\documentclass[12pt,a4paper]{article}
\usepackage[utf8]{inputenc}
\usepackage[T1]{fontenc}
\usepackage[french]{babel}
\usepackage{graphicx}
\usepackage{geometry}
\geometry{hmargin=2.5cm,vmargin=2.5cm}
\usepackage{array}
\usepackage{multirow}
\usepackage{mathtools, bm}
\usepackage{amssymb, bm}
\usepackage{amsmath,amsfonts,amssymb}
\usepackage{lmodern}
\usepackage{amsmath}
\usepackage{array}
\usepackage{multirow}


\title{\textbf{Rapport de Statistique Appliquée}}
\author{Brian Whannou, Alexis Toullet, Eric Nana Njoya, David Billod de Filiquier\\
Encadrants : L.Linnemer, M.Visser}

\date{\today}

\begin{document}

\begin{titlepage}

\maketitle

\vspace{5cm}

\hrulefill
\begin{center}
 \begin{huge}
\textbf{Etude du marché du viager}\\
\vspace{1cm}
 - Vendre ou ne pas vendre, telle est la question - 
 \end{huge}
 \end{center} 

\hrulefill


\end{titlepage}


\section*{Introduction}

\paragraph{}
Comment un retraité dont le patrimoine se limite à sa résidence principale peut-il rester chez lui tout en augmentant ses revenus ? Une solution réside dans la vente du bien en viager. En effet celle-ci permet de rendre liquide une partie de la valeur du bien immobilier. D’après le Code Civil, le viager est un contrat aléatoire, c’est-à-dire qu’un des deux contractants gagnera ou perdra un certain montant, indéterminé à la signature du contrat. Si cet aléa n’existe pas, le contrat n’est pas valide.
Au départ, le viager permet au vendeur qui était propriétaire du logement avant la signature du contrat de continuer à vivre dans ce logement même si le titre de propriété appartient alors à l’acheteur qui verse au vendeur un bouquet (capital initial) à la signature du contrat et une rente périodiquement. Le contrat prend fin à la mort de l’un des deux contractants, en principe le vendeur. Cependant de nombreux aménagements sont possibles tant qu’ils ne remettent pas en cause l’aléa. Le vendeur peut aussi quitter son logement, on parle alors de viager libre. D’après le Centre Européen de viagers, le viager est effectivement libre dans 5\% des cas. D’après le Conseil économique et social, il existe trois autres catégories de viager : les viagers sans rente (ou nue-propriété), les viagers dont l’occupation est limitée dans le temps, les viagers occupés dont le paiement de la rente est limité dans le temps.
\paragraph{}
A la signature du contrat, le vendeur verse éventuellement à l’acheteur un bouquet qui vaut habituellement 20 à 30\% de la valeur du bien qui est estimée par un notaire mais peut atteindre 50\% du prix du bien, si ce n’est plus. Les Code Civil et Fiscal stipulent que « le prix de vente d’un immeuble aliéné en viager doit représenter la valeur du bien ». En outre, ce prix doit être exprimé dans le contrat et converti en rente viagère. Cependant la rente n’est pas calculée en prenant simplement en compte le prix du bien : la durée de vie du vendeur entre aussi en jeu. D’après le Conseil économique et social, la valeur du bien, le montant du bouquet éventuel, la rentabilité du bien, la réversibilité possible de la rente, l’âge du crédirentier sont aussi pris en considération pour fixer la rente. En général, la rente est fixée librement même si l’administration fiscale peut intervenir si le montant est jugé trop faible. La plupart des contrats indexe aussi la rente sur l’indice des prix à la consommation (fourni par exemple par l’INSEE) : cela garantit que le revenu réel du crédirentier reste constant dans le temps.
\paragraph{}
 En principe, les vendeurs sont des personnes âgées, essentiellement veuves ou célibataires âgés sans héritiers proches. La plupart des contrats concernent des biens immobiliers de Paris et sa banlieue et dans les grandes villes de Provence-Alpes-Côte d’Azur (Cannes, Menton, Nice etc.). D’après Griffon, seulement 4000 contrats en viager sont signés par an. Malgré les incitations des pouvoirs publics, le marché du viager en France demeure restreint. Une explication naturelle réside dans l’éventuelle présence d’asymétrie d’information comme l’illustre le cas de Jeanne Calmant. En 1965, Jeanne Calmant nonagénaire et sans héritier vend sa maison en viager à son notaire alors âgé de 47 ans. Or Jeanne Calmant s’éteint à 122 ans. Son notaire, qui est décédé vingt ans avant elle, lui a versé le double de la valeur de la maison. Pour rassurer les craintes à propos de la durée a priori aléatoire du versement de la rente et répondre aux attentes des deux parties du contrat, les professionnels de l’immobilier peuvent utiliser des tables de durée de vie probables construites à partir des  tables de mortalité  établies par l’INSEE qui indiquent  « le quotient de mortalité des hommes et des femmes à chaque âge pour une période donnée, ainsi que l’espérance de vie qui en résulte ». L’agence immobilière Legasse Viager qui nous a aimablement fourni la base de données utilise quant à elle la table de mortalité dite de Daubry tandis que d’autres agences préfèrent construire leur propre table basée sur la survie de leurs précédents clients.
\paragraph{}
Malgré la méfiance de l’opinion d’activité, le viager est un secteur d’avenir qui mérite donc d’être étudié. En 2013, 1\% seulement des transactions immobilières concernaient des biens en viager mais le 9 septembre 2014 \footnote{Source: caisse des dépôts}, la Caisse des dépôts lance le projet Certivia afin de dynamiser le marché du viager qu’elle juge encore insuffisamment développé. Elle estime en outre que ce marché permet de répondre aux enjeux du vieillissement de la population. Selon l’Insee, en 2040, 30\% de la population française, c’est-à-dire 8 millions de personnes, devraient avoir plus de 60 ans. D’après l’Insee, 72\% des personnes de plus de 70 ans sont propriétaires de leur résidence principale et la Caisse des dépôts juge  « qu’un nombre croissant d’entre elles connaitra une diminution relative de leurs revenus ». Certivia fait office d’acheteur d’un bien qui doit être occupé : l’achat en viager de biens  libres ne fait pas partie de ses attributions et  le vendeur, obligatoirement un particulier, doit avoir plus de 70 ans. 
\paragraph{}
Il est ainsi important pour la Caisse des dépôts d’avoir à sa disposition des tables de mortalité fiables : les tables de mortalité permettent certes d’estimer l’âge de décès d’un individu quelconque mais sont-elles véritablement utilisables pour les vendeurs en viager ? Ces derniers ne sont-ils pas atypiques en termes d’espérance de vie ? Ne faudrait-il pas créer une table de mortalité spécifique pour ces derniers ?  Pour répondre à ces questions nous disposons d’une base de données fournie par nos encadrants qui répertorient 564 transactions viagères conclues via l’agence immobilière parisienne Legasse Viager, et contient diverses informations sur les vendeurs comme leur sexe, leur âge au décès, leur âge au moment de signer le contrat, le montant de la rente, etc. A partir de cette base dont l’étude statistique sera menée plus loin, nous avons cherché à mettre en lumière les caractéristiques des vendeurs en viager. Pour cela, la base comprend, pour chaque vendeur, un clone moyen, c’est-à-dire une personne du même sexe que le vendeur, née le même mois de la même année et dans le même département. Nous possédons des informations sur ces clones, comme leur durée de vie, ce qui va permettre d’étudier les différences potentielles existant entre clones et vendeurs en terme de durée de vie, c’est-à-dire comprendre la spécificité des vendeurs en viager par rapport au reste de la population. 
Nous allons commencer par exposer les principes théoriques qui sous-tendent notre étude. Puis, à l’aide de statistiques descriptives, nous allons montrer la composition de la base de données, afin de mieux comprendre quelles sont les personnes inscrites dans cette base. Enfin, nous détaillons notre méthode pour quantifier les différences entre vendeurs en viager et clones ainsi que les résultats issus de cette méthode.

\newpage

\section{FONDEMENTS THÉORIQUES}
\subsection{Microéconomie}
\paragraph{}
Dans l’article \textit{Testing for asymetric information in the viager marke} publié en 2012 au \textit{Journal of Public Economics},P.  Février,  L. Linnemer et M. Visser s’interrogent sur la présence d’asymétrie d’information sur le marché du viager en utilisant des données notariales provenant de transactions ayant eu lieu à Paris et dans sa banlieue entre 1993 et 2001. Pour chacune des transactions sont observées le bouquet, la rente, le prix de marché du bien immobilier et quelques caractéristiques des acheteurs et des vendeurs. Les dates de décès des vendeurs ne sont pas fournies dans cette base de données.

\paragraph{}

L’approche de ces trois auteurs est fondée sur la théorie microéconomique des contrats et postule que le type du crédirentier peut être estimé. Soit $\pi_{t}$ la probabilité que le crédirentier meurt exactement t périodes après avoir signé le contrat pour t $\in\left\llbracket 0;T\right\rrbracket$. Soit $\delta$ le facteur d’escompte et $r=\frac{1}{\delta}-1$ le taux associé (les auteurs l’ont pris égal à 0.05). Le type d’un individu quelconque s’écrit alors : $\alpha=\sum_{t=0}^{T}\delta^{t}\pi_{t}$. Le type du vendeur décroît donc avec son espérance de vie et peut alors être calculé via les tables de mortalité.

\paragraph{}

La condition de non arbitrage est essentielle dans l’analyse menée par les auteurs. Elle stipule que les acheteurs sont indifférents entre l’achat d’un bien immobilier en viager et d’une bien sur le marché standard. Cette condition implique aussi que les débirentiers soient neutres au risque ce qui est aussi bien le cas d’après les tests menés par les auteurs.Le bouquet B est versé à la signature du contrat à t=0 et les rentes R au début de chaque période $t=1,2,\ldots$ jusqu’à la mort du crédirentier. On note V la valeur du bien immobilier. La condition de non arbitrage permet alors d’écrire : $\alpha V=B+\frac{1-\alpha}{r}R$. D’où on déduit $\alpha=\frac{\nicefrac{rB}{V}+\nicefrac{R}{V}}{r+\nicefrac{R}{V}}$. Les paramètres du contrat permettent donc de calculer le type du vendeur sans passer par les tables de mortalité. 

\paragraph{}

Pour comparer les probabilités de survie des débirentiers et d’un agent quelconque, les auteurs se sont donc intéressés à leur type. Pour un agent quelconque, le type est calculé à l’aide des tables de mortalité de l’INSEE. Mais comme auteurs mettent en doute la fiabilité de ces tables pour les vendeurs en viagers, elles ne sont pas utilisées pour estimer le type des vendeurs : puisque l’étude montre qu’on peut accepter l’hypothèse de non-arbitrage, les types des vendeurs en viager sont calculés à l’aide des paramètres du contrat. Sur l’échantillon de 874 individus, on obtient alors $\hat{\alpha}=0.7$, l’écart-type valant 0.13. En outre autant l’âge et le sexe expliquent particulièrement bien le type d’un agent quelconque $\left(R^{2}=0.98\right)$, autant ça n’est pas le cas pour le type des crédirentiers $\left(R^{2}=0.11\right)$,. En outre les crédirentiers ont un type généralement plus grand que les individus comparables de la population globale ce qui signifie que \textbf{l’espérance de vie d’un vendeur en viager est plus faible que celle d’un individu quelconque}. Cela peut s’expliquer par le fait que les crédirentiers sont relativement plus pauvres que les individus comparables dans la population.

\paragraph{}

L’identifiabilité du type permet aussi de s’interroger sur la transmission de l’information à propos des probabilités de survie : la connaissance des probabilités de survie par les deux parties du contrat est-elle symétrique ou asymétrique? Si l’information est symétrique, les deux parties ont les mêmes connaissances sur ces probabilités : le crédirentier révèle son type lorsqu’il contacte le débirentier, le rencontre et lui fait visiter le bien immobilier. En cas d’asymétrie, le crédirentier peut signaler son type à l’aide des paramètres du contrat. Les tests indiquent qu’on peut rejeter l’asymétrie d’information. Ainsi le crédirentier est capable de transmettre de manière crédible l’information. De plus, d’autres études montrent que chaque agent connaît son propre type.

\subsection{Outils statistiques}

Les temps de survie ont deux caractéristiques importantes par rapport à d’autres données. Tout d’abord puisque les temps de survie sont positifs, l’hypothèse de normalité n’est pas vraisemblable. De plus, certaines données sont censurées, c’est-à-dire que pour certains individus l’événement observé n’a pas lieu pendant la période d’observation.

\subsubsection{Troncature et censure}

\paragraph{}

Il faut ici mettre en lumière un problème inhérent à toute enquête ou relevé de donnée ayant attrait à certaines données temporelles : la censure et la troncature. Ainsi si on demande à une personne depuis quand elle est au chômage et qu’elle répond être sans emploi, depuis 3 semaines, alors on ne peut pas conclure qu’elle a été au chômage depuis 3 semaines, mais seulement qu’elle a été sans emploi au moins 3 semaines. 

\paragraph{}

On affronte dans notre étude le problème de la troncature des données. En effet on n’observe seulement les transactions ayant eu lieu avant Octobre 2009, date de fin de l’intervalle de temps étudié. Au-delà nous ne disposons d’aucune donnée, d’où le problème de la troncature qui induit un éventuel biais de sélection. Il faut alors tenir compte de cette limite concernant nos données, sans quoi nous faussons nos estimations. Il faut alors incorporer cet élément dans notre étude, ce qui sera fait et expliqué lorsque l’on évoquera la construction de la vraisemblance, un peu plus loin. 

\subsubsection{Fonction de risque}

Soit T la durée de vie, variable aléatoire définie sur $\left[0,+\infty\right]$ de fonction de répartition F, de densité f. On définit alors la fonction de survie S par : $S\left(t\right)=1-F\left(t\right)$. S est donc continue décroissante et telle que $S\left(0\right)=1$ et $\underset{+\infty}{lim}S=0$. On obtient alors immédiatement $f\left(t\right)=-\frac{dS\left(t\right)}{dt}$.
On définit ensuite la fonction de risque $\lambda$ sur $\left[0,+\infty\right]$ par :

$\lambda\left(t\right)=\underset{\triangle t\rightarrow0}{lim}\frac{\mathbb{P}\left(t\leq T\prec t+\triangle t\mid T\geq t\right)}{\triangle t}$

La probabilité au numérateur est celle de décès d’un individu durant l’intervalle $\left[t;t+\triangle t\right]$ sachant qu’il est vivant en t. C’est une mesure du risque instantané. Cela n’est pas une probabilité : elle peut être supérieure à 1. Si la dérivée de la fonction de risque est positive, on parle de dépendance positive à la durée.

La formule de Bayes donne ensuite : $\lambda\left(t\right)=\frac{f\left(t\right)}{S\left(t\right)}=-\frac{dlogS\left(t\right)}{dt}$. Enfin, on définir le risque cumulé \varLambda sur $\left[0,+\infty\right]$ par : $\varLambda\left(t\right)=\int_{0}^{t}\lambda\left(s\right)ds$. On déduit alors $S\left(t\right)=\exp\left(-\varLambda\left(t\right)\right)$.

La connaissance d’une des trois fonctions $\left(S,f,\lambda\right)$
  nous permet de déduire les deux autres à partir des relations précédentes. Dans l’article \textit{Economic Duration Data and Hazard Functions} N. Kiefer présente trois distributions particulières :

• Distribution exponentielle : la fonction de risque est alors constante, égale au paramètre de la loi. Il n’y a donc aucune dépendance à la durée : on parle de modèle sans mémoire.

• Distribution de Weibull de paramètres strictement positifs $\alpha$
  et $\lambda$
 , de densité $f\left(t\right)=\gamma\alpha t^{\alpha-1}\exp\left(-\gamma t^{\alpha}\right)$
  et de fonction de risque $\lambda\left(t\right)=\gamma\alpha t^{\alpha-1}$
 . Si $\alpha=1$
 , c’est la distribution exponentielle. Si $\alpha\prec1$, la fonction de risqué est décroissante. Dans le cas contraire, elle est croissante.

• Distribution log-logistique de paramètres strictement positifs \alpha
  et $\lambda$, de densité $f\left(t\right)=\nicefrac{\left(\gamma\alpha t^{\alpha-1}\right)}{\left(1+t^{\alpha}\gamma\right)^{2}}$
 et de fonction de risque $\lambda\left(t\right)=\nicefrac{\left(\gamma\alpha t^{\alpha-1}\right)}{\left(1+t^{\alpha}\gamma\right)}$.

\paragraph{}

Il apparaît naturel de commencer avec une fonction de risque constante, ce qui correspond à une distribution de probabilité exponentielle de paramètre ce taux constant. Cette loi va particulièrement bien s’adapter si l’échantillon contient des durées de survie qui ne varient pas beaucoup. De plus on remarque que l’espérance et la variance ne peuvent être ajustées séparément car elles dépendent toutes les deux d’un seul paramètre, donc on comprend bien qu’un échantillon contenant des durées de survie variées ne sera probablement pas très bien modélisé par une loi exponentielle. Pour pallier ce problème, on peut se tourner vers une forme de loi plus générale, avec la loi de Weibull, ayant deux paramètres et dont la loi exponentielle n’est qu’un cas particulier. L’inconvénient d’un tel choix de modélisation est d’avoir un hasard strictement monotone. On peut alors se tourner vers une modélisation par la loi log-logistique qui permet d’avoir un hasard non strictement monotone.

\subsubsection{Méthodes d'estimation}

\subsubsubsection{Variable explicative dans les modèles à risque proportionnel}

Le modèle à hasard proportionnel est populaire car aisé à interpréter. Les variables explicatives ont pour effet de multiplier le hasard par un facteur d’échelle. Malheureusement les coefficients associés aux variables explicatifs ne peuvent s’interpréter aussi facilement que dans le cas d’un modèle de régression linéaire, car ils ne sont pas le fruit de la dérivée partielle du hasard par rapport à une variable. 

\paragraph{}

Toutefois, on peut se ramener à une interprétation plus aisée des coefficients lorsque l’on suppose un risque de la forme : 

$\lambda\left(t,x,\beta,\lambda_{0}\right)=\varphi\left(x,\beta\right)\lambda_{0}\left(t\right)$, $\lambda_{0}$ étant un risque choisi arbitrairement.

Notons que le paramètre $\theta$ s’écrit maintenant $\varphi\left(x,\lambda_{0}\right)$. Le facteur $\varphi$ ne dépend pas de la durée. Généralement, on pose $\varphi\left(x,\beta\right)=\exp\left(x^{'}\beta\right)$. Cette modélisation est pratique car elle n’impose aucune restriction sur le signe de $\beta$. Comme nous allons le voir, l’estimation de $\beta$ ne dépend pas du choix de $\lambda_{0}$. En effet,$\frac{\partial\ln\lambda\left(t,x,\beta,\lambda_{0}\right)}{\partial x}=\frac{\partial\ln\varphi\left(x,\beta\right)}{\partial x}$. L’effet proportionnel de x sur le risque ne dépend donc pas de la durée.

\subsubsubsection{Estimateur du maximum de vraisemblance avec des données tronquées}

On se place ici dans le cadre où l’on suppose connaître la forme de la densité de probabilité de la durée. L’étude revient alors à estimer au mieux le paramètre dont dépend la distribution. En notant θ le paramètre à estimer, la fonction de vraisemblance L s’écrit habituellement : 

$L^{*}\left(\theta\right)=\stackrel[k=1]{n}{\prod}f\left(t_{k},\theta\right)$
 

Ici les données sont partiellement tronquées. Soit d_{i}
  l’indicatrice qui vaut 1 lorsque la $i^{^{\grave{e}me}}$
  donnée a été censurée. La log-vraisemblance l s’écrit alors : A ETOFFER

$l\left(\theta\right)=\stackrel[i=1]{n}{\sum}\left(\ln\left(f\left(t_{i},\theta\right)-\ln\left(1-S\left(t_{i},\theta\right)\right)\right)\right)$
 

On maximise alors la log-vraisemblance de manière usuelle. On peut alors construire des tests et des intervalles de confiance. Il est à noter que si l’on ignore le problème de censure, alors on aurait un biais positif pour l’estimation du paramètre. Sous certaines conditions, l’estimateur du maximum de vraisemblance est consistent.

\section{DESCRIPTION DE LA BASE DE DONN\'EES}


\end{document}